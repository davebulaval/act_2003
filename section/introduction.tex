\chapter{Introduction}
L'établissement de prévisions joue un rôle central dans notre vie de tous les jours (prévisions météorologique, horoscope, etc.), et plus particulièrement dans celle des actuaires.

\subsection*{Deux grandes classes de prévisions :}
\begin{itemize}
\item Qualitatives : basées sur des opinions et/ou des intuitions.
\item Quantitatives : basées sur des observations, un modèle et des arguments mathématiques.
\end{itemize}

\subsection*{Deux \textit{grandes étapes} pour établir des prévisions quantitatives}
\begin{enumerate}
\item Bâtir le modèle :
\begin{itemize}
\item[ex:] $F = M \times a$ Qui représente un modèle déterministe
\item[ex:] $Y =3 \times X + 6 + \epsilon_t \text{ ;où} \epsilon_t \sim N(0, 10)$ Qui représente un modèle probabiliste 
\end{itemize}
\item Calculer les prévisions à partir du modèle.
\end{enumerate}

\bigskip
Dans le cadre du cours, seulement les modèles probabilistes linéaires seront étudiez. 
